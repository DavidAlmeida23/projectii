
Over the code analysis and modelization of the \textit{DBT} component, the bindings and interfaces established with the Decoder became evident. By inspection, it is possible to see that this component has an interface with the following components: Source Architecture, Source Environment, CCache, Generator and Translator. Next, each binding will be covered one-by-one, being revealed its importance as well as the methods that composes it.

\subsubsection*{Decoder - Source Architecture}
    The Source Architecture is the component that only provides services. The reason for that is based on the fact that its essence consists in the specification of the contents of an architecture. Figure \ref{fig:DecoderBind1}, there are \textbf{two interfaces} between the Decoder and the Source Architecture components (figure \ref{fig:DecoderBind1}).
    
    \begin{figure}[!htb]
    \centerline{
    \includegraphics[scale=0.45]{images/DecoderBind1}
    }
    \caption{Decoder - Source Architecture bindings.}
    \label{fig:DecoderBind1} 
    \end{figure}
    
    The first binding (\textbf{R4\_1 - S4}) lies in an interface which has a function that returns the number of bits of the opcode (\texttt{nBitsOpcode}). This value is used in the generation of the final sources by the elaboration files. In the second binding (\textbf{R6\_1 - S6}), the interface \texttt{Registers} is recognized as an empty interface, \textit{i.e.}, exists and must be respected but there is no way of define functions on it because each architecture has its own registers. In the source files, the definitions of the registers is found in an header file as macros.


\subsubsection*{Decoder - Source Environment}
    As the name suggests, the Source Environment is associated to important elements of the execution environment such as the \textit{PC} and the data memory. 
    
    \begin{figure}[!htb]
    \centerline{
    \includegraphics[scale=0.45]{images/DecoderBind2}
    }
    \caption{Decoder - Source Environment binding.}
    \label{fig:DecoderBind2} 
    \end{figure}
    
    For the Decoder, the value stored \textit{PC} is necessary in decoding branch instructions. Thereby, one of the services that the Source Environment component must provide is the \textit{PC} access which translates in to an interface whose methods must allow get and set the \textit{PC} value, \texttt{getPC} and \texttt{setPC} (figure \ref{fig:DecoderBind2}). 
     
    \subsubsection*{Decoder - CCache}
    In order to decode the operands of an instruction, it is necessary in a prime instance to fetch it. This step is made inside the decoder since the length of the instruction depends of the source ISA, and so, there are situation where this value is not constant. For those situation, the instructions must be fist detected and after fetched. Figure \ref{fig:DecoderBind3} shows the binding established.
    
    \begin{figure}[!htb]
    \centerline{
    \includegraphics[scale=0.45]{images/DecoderBind3}
    }
    \caption{Decoder - CCcache binding.}
    \label{fig:DecoderBind3} 
    \end{figure}
    
    As has been said before, CCache stores portions of binary source code that needs to be translated. Thus, an interface is established and must contain a function that accesses the memory and returns an instruction (\texttt{fetch}). 
       
    \subsubsection*{Decoder - Generator}
    
    Among other responsibilities, the decoder decides which methods must be called for target code generation. This generation service is provided by the Generator which contains the implementations of the generation functions for the target architecture.
	That being said, an interface between the Decoder and the Generator is visible and consists in the \texttt{gen\_x} functions that were mentioned before. Examples of \texttt{gen\_x} functions are shown in figure \ref{fig:DecoderBind4}.
    
    \begin{figure}[!htb]
    \centerline{
    \includegraphics[scale=0.45]{images/DecoderBind4}
    }
    \caption{Decoder - Generator binding.}
    \label{fig:DecoderBind4} 
    \end{figure}
    
    
\subsubsection*{Decoder - DBT Engine}   
    The last service referenced by the Decoder is provided by the DBT Engine (figure \ref{fig:DecoderBind6}). It is composed internal variables of the engine, such as, \texttt{eoBB} (signals the end of a basic block, and so, the end of translation), \texttt{eoExec} (signals the end of execution) and at last \texttt{pSourceProgMem} (pointer to code memory used in instructions that access this the contents of it).
    
	\begin{figure}[!htb]
    \centerline{
    \includegraphics[scale=0.45]{images/DecodeBind6}
    }
    \caption{Decoder - DBT Engine binding.}
    \label{fig:DecoderBind6} 
    \end{figure}
    

\subsubsection*{Decoder - Translator}

	Until now, all the bindings of the Decoder were references to services of other components. But, concerning the correct operation of the DBT engine, the decoder must step in and provide a decoding service to the translator (figure \ref{fig:DecoderBind5}).
    
	\begin{figure}[!htb]
    \centerline{
    \includegraphics[scale=0.45]{images/DecoderBinding5}
    }
    \caption{Decoder - Translator binding.}
    \label{fig:DecoderBind5} 
    \end{figure}
    
    A connection between the Decoder and the Translator is established, consisting in an interface that contains a \texttt{decode} function. So, when a dynamic binary translator is created using the reference architecture that is being design, an implementation of a decode function must inevitably exist.

