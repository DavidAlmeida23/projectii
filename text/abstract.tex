The use of Domain Specific Languages (DSL) can result in high productivity and high quality design in the generation of final configurable projects. The current project covers the process of modeling an embedded system, more specifically a Dynamic Binary Translator, through the use of a new DSL called EL language, in order to obtain a configurable system. 

The work done results in the design of a conceptual model and a implementation that validates the use of these tools. The model created was based on a Service Component Architecture (SCA) that allows the definitions of computational entities described as components. Components can be arrange into composites and linked through services and references. This allows to describe complex architecture composed by replaceable modules.  

Using a modeling framework, the \textit{EL} framework, it was possible to create a representation of the model in code and from there achieve a goal where it is possible to configure some particular parameters of the system and generate final source files of a Dynamic Binary Translator. 