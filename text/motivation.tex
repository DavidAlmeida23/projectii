
Given the growing need for automation of complex embedded systems, the design and use of a {\textbf{Domain Specific Language} \textbf{(\textit{DSL})} has an important role in achieving high productivity and high design quality in the generation of final configurable projects. A \textit{DSL} is a programming language which facilitates the expressing of semantics suitable for a specific domain, and so it can be used for modeling embedded system's software. 

To address this problem, this work has as its main goal the use of a developed \textit{DSL}, called \textbf{Elaboration Language (\textit{EL})}, to model a \textbf{reference architecture} for the generation of a final system. The embedded system in study is a \textbf{Dynamic Binary Translator} \textbf{(\textit{DBT})}.

The developed framework is capable of reading \textbf{\textit{.el}} files and generate final software systems. This framework was developed in collaboration with the whole class of Embedded Systems' course, by so, this work describes it briefly and gives more focus to the use of the language to model the \textit{DBT} and generate a system configurable by a user. The \textit{DBT} was assigned to four groups of the Project II course, each one responsible for model one or more components of the translator, namely, the Source and Target Architectures, the Memory structures, the Decoder, and at last, the Generator.  
There are plenty of possible implementations  for these components, and so, several models will be presented, all with different behaviors and specific interfaces with other components.
