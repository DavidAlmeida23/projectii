The requirement elicitation phase is certainly relevant in the definition of the project since it allows to specify its functional needs. There are two types of requirements defined next: \textbf{Functional Requirements} and \textbf{Non-Functional Requirements}. 

\subsubsection{Functional}
\begin{itemize}
\item Provides a \textbf{model for a Dynamic Binary Translator}, showing its components and functionalities.
\item Provides \textbf{several models for the Decoder component}, each one represented by its own behavior and interfaces with other components. 
\item Allows the user to \textbf{configure parameters based on his preferences} for the system, which will be transposed to generated code.
\item Provides\textbf{ code generation of final project} (software and hardware implementation) based on the conceptual models created.
\end{itemize}

\subsubsection{Non-Functional}
\begin{itemize}
\item A robust and well defined model of the DBT and Decoding Operands shall be accomplished.
\item The generated code must be compatible with the target application (DBT), allowing possible hardware migrations of functionalities and automation achieving.
\end{itemize}

